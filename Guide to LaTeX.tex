\documentclass[11pt]{article}
\usepackage{amsmath,amsfonts, color, booktabs, centernot, textcomp,amssymb,geometry,graphicx,verbatim,enumerate, bbm}
\usepackage{fancyhdr}
\pagestyle{fancy}
\setlength\parindent{0pt}
\newcounter{sectionnum}
\def\Name{Leah Dickstein}  % Your name
\def\Homework{n}% Homework number
\def\Session{Spring 2015}


\title{Guide to LaTeX}
\author{\Name}
\lhead{Guide to LaTeX: Section \thesectionnum}
\rhead{\Name}


\begin{document}
\maketitle

\textbf{Note}: Make sure you have LaTeX and a good TeX editor!\\
Desktop:
\begin{itemize}
\item TexMaker (what I personally use)
\item LyX (a WYSIWYG, meaning it's basically Word for LaTeX. Lower learning curve, but I think it's slower in the long run because the whole point is NOT using your mouse.)
\end{itemize}
Browser:
\begin{itemize}
\item ShareLaTeX (comes with a ton of templates!)
\item Overleaf (compiles real-time, but is slower)
\end{itemize}

\textbf{Note 2:} Don't pronounce it Latex like the plastic. It's either Law-Tech or Lay-Tech. I prefer the former.

\section{Getting Started}
\setcounter{sectionnum}{1}
We've already provided you with HW solution templates and self-grade templates, which should allow you to skip past this section. If you're starting from scratch, you start your blank document with:

\begin{verbatim}
\documentclass[11pt]{article}
\begin{document}
Body of text.
\end{document}
\end{verbatim}

Anything between \verb+\documentclass+ and \verb+\begin{document}+ is called the \textbf{Preamble} of the article, and anything below is the \textbf{Body}. Anything below \verb+\end{document}+ won't be compiled, so you can put temporary text or text that contains compile errors for storage.\\

Alright, it's time to begin: with a title! In the preamble, put:
\begin{verbatim}
\title{Guide to LaTeX}
\author{Your Name Here!}
\end{verbatim}

LaTeX will automatically put the date it was last compiled. Now you're good to go, and you can just start typing. However, throwing a bunch of text on a page is pretty hard to read...two things:\\

1) LaTeX ignores whitespace. Start newlines with either \verb+\\+ or \verb+\newline+.\\

2) Give your document more structure with sections! LaTeX has quite a few options:

\part{An Example Part}
The command: \verb+\part{An Example Part}+\\

\chapter{An Example Chapter}\\
The command: \verb+\chapter{An Example Chapter}+\\
Note that the other section headers besides Chapter have an auto-newline after, but if you want a newline after Chapter you need to do it yourself.

\section{An Example Section}
The command: \verb+\section{An Example Section}+\\

\subsection{You can have subsections}
The command: \verb+\subsection{You can have subsections}+\\

\subsubsection{And subsubsections!}
The command: \verb+\subsubsection{And subsubsections!}+\\
That's the smallest you can go :( This isn't inception.\\

LaTeX auto-numbers everything, but if you don't want the numbering just add an asterisk. \verb+\section{Unicorns!}+ $>>$ \verb+\section*{and Rainbows!}+\\

\textcolor{blue}{\textbf{For the purposes of your homework}}, ignore parts and chapters. Just use sections.

\setcounter{section}{1}
\section{Environments}
\setcounter{sectionnum}{2}

\begin{itemize}
\item inline math
\item outline math
\item itemize
\item enumerate
\item verbatim
\end{itemize}

\section{More Formatting Magic}
\subsection{Table of Contents}

\subsection{Margins}

\subsection{Minipage}
Sometimes you want to have multiple columns, or to split your page up into sections. Enter the minipage command. If I want to have two columns, I would just\\

\begin{minipage}{0.5\textwidth}
use the command
\end{minipage}
\begin{minipage}{0.5\textwidth}
\begin{verbatim}
\begin{minipage}{0.5\textwidth}
text here!
\end{minipage}
\end{verbatim}
\end{minipage}

\section{Useful Packages}
\subsection{Links}

\subsection{Colors}
Add to the header of the document \verb+\usepackage{color}+ to gain access to the default 8 colors. Alternatively, you can write \verb+\usepackage[usenames,dvipsnames,svgnames,table]{xcolor}+, which will give you access to a BAJILLION colors.

\subsection{Algorithms}

\subsection{Circuits}

\section{Appendices}

\end{document}
